	\section{Introduction}

When Arthur Compton investigated the scattering of X-rays by electrons in 1923, he encountered a result that the classical wave theory of light could not explain. The wavelength of the scattered radiation was greater than that of the incident beam — and the difference depended on the scattering angle. This experiment provided the first direct evidence for the transfer of momentum between photon and electron, thereby confirming the particle nature of light.

The Compton effect marks a turning point in physics: it links energy and momentum conservation on a quantum-mechanical level and leads to the well-known \textbf{Compton formula}, which precisely describes the wavelength shift. The following derivation develops this relationship step by step — historically framed, mathematically sound, and physically transparent.

\vspace{1.2em}
\begin{quote}
	\small
	\textbf{Note on Methodology.}  
	This paper was created in close collaboration between human and artificial intelligence. 
	The physical reasoning and overall structure originate entirely from the author; the AI served as a tool for formulation, refinement, and didactic clarity. 
	The aim is to demonstrate that modern AI assistance does not replace human scientific work but enhances it — through greater clarity, structure, and transparency.
\end{quote}
\vspace{1.2em}