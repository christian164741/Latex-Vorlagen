% =============================================================
% Paper – „Herleitungen im historischen Kontext“
% Autor: Christian Weilharter
% Serie: Theoretische Physik – Herleitungen
% Ziel: 6×9 in (KDP/Print) + EPUB-kompatibles PDF (Apple Books)
% =============================================================

\documentclass[12pt,oneside]{scrartcl}

% ---- Sprache & Kodierung ----
\usepackage[utf8]{inputenc}
\usepackage[T1]{fontenc}
\usepackage[ngerman]{babel}
% ---- Literatur (BibLaTeX/Biber) ----
\usepackage[backend=biber,style=numeric,sorting=nyt]{biblatex}
\renewcommand*{\bibfont}{\small}
\addbibresource{../bib/literatur.bib} 
% ---- Gemeinsamer Stil ----
\usepackage{../styles/paper-style-de}

% =============================================================
% Serien- und Publikationsmetadaten
% =============================================================
\newcommand{\SeriesTitle}{Theoretische Physik – Herleitungen im historischen Kontext}
\newcommand{\PaperTitle}{Der Compton-Effekt – Herleitung, Geschichte und Anwendungen}
\newcommand{\PaperAuthor}{Christian Weilharter}
\newcommand{\PaperDate}{26.\ Oktober\ 2025}

\newcommand{\PaperDOI}{\emph{DOI (Zenodo)}: \texttt{10.5281/zenodo.xxxxxxx}}
\newcommand{\PaperISBNPDF}{\emph{ISBN (Print)}: 978-3-xxxx-xxxx-x}
\newcommand{\PaperISBNEPUB}{\emph{ISBN (E-Book)}: 978-3-xxxx-xxxx-x}
\newcommand{\PaperKeywords}{Compton-Effekt; Quantenphysik; Röntgenstreuung; Photonenimpuls; Vierervektoren}

% ---------- Metadatenblock ----------
\newcommand{\PaperMetadata}{
	\vspace{1em}
	\begin{flushleft}
		\small
		\setlength{\parskip}{3pt}
		\noindent\PaperDOI\\
		\PaperISBNPDF\\
		\PaperISBNEPUB\\
		\textit{Schlagwörter:} \PaperKeywords
	\end{flushleft}
	\vspace{1em}
}

% =============================================================
% Titel
% =============================================================
\title{\PaperTitle\\[0.5em]\large \SeriesTitle}
\author{\PaperAuthor}
\date{\PaperDate}

% =============================================================
% Dokument
% =============================================================
\begin{document}
	
	\thispagestyle{empty}
	\begin{center}
		{\Huge\bfseries \PaperTitle\par}
		\vspace{0.5em}
		{\Large\itshape \SeriesTitle\par}
		\vspace{2em}
		{\Large \PaperAuthor\par}
		\vspace{0.5em}
		{\large \PaperDate\par}
	\end{center}
	
	\PaperMetadata
	
	\vspace{1em}
	\hrule
	\vspace{1em}
	
	\noindent\textbf{Kurzfassung.}  
	Dieses Paper verbindet eine kompakte Herleitung der Compton-Formel mit dem historischen Kontext von Planck bis Compton und skizziert Anwendungen in der modernen Physik. Der Fokus liegt auf Verständlichkeit, mathematischer Präzision und der Einbettung in die Wissenschaftsgeschichte.
	
	\vspace{1em}
	\hrule
	\vspace{2em}
	
	% =============================================================
	% Inhalt
	% =============================================================
	% Hauptkapitel
		\section{Introduction}

When Arthur Compton investigated the scattering of X-rays by electrons in 1923, he encountered a result that the classical wave theory of light could not explain. The wavelength of the scattered radiation was greater than that of the incident beam — and the difference depended on the scattering angle. This experiment provided the first direct evidence for the transfer of momentum between photon and electron, thereby confirming the particle nature of light.

The Compton effect marks a turning point in physics: it links energy and momentum conservation on a quantum-mechanical level and leads to the well-known \textbf{Compton formula}, which precisely describes the wavelength shift. The following derivation develops this relationship step by step — historically framed, mathematically sound, and physically transparent.

\vspace{1.2em}
\begin{quote}
	\small
	\textbf{Note on Methodology.}  
	This paper was created in close collaboration between human and artificial intelligence. 
	The physical reasoning and overall structure originate entirely from the author; the AI served as a tool for formulation, refinement, and didactic clarity. 
	The aim is to demonstrate that modern AI assistance does not replace human scientific work but enhances it — through greater clarity, structure, and transparency.
\end{quote}
\vspace{1.2em}
		\section{Historical Context – The Road to the Compton Effect}

\subsection{Initial Situation: Light as a Wave}

At the beginning of the 20th century, light was considered, according to the theory of \textbf{James Clerk Maxwell} (1860s), to be an electromagnetic wave. 
This model successfully explained interference, diffraction, and polarization – and appeared to be complete.


	\section{Derivation of the Compton Formula}
Energy and momentum conservation before and after the collision, elimination of the electron velocity, and conversion to wavelength form
	\chapter{Experimentelle Bestätigung des Photons}
\setcounter{section}{4}
\setcounter{subsection}{0}
\setcounter{subsubsection}{1}
\setcounter{secnumdepth}{3}
\setlength{\parindent}{0pt}
% Boxen-Stile definieren

\subsection{Der Photoeffekt }\index{Photoeffekt}

\subsubsection{ Einleitung und klassische Erwartung}\index{Klassische Elektrodynamik}\index{Wellenmodell}

Der sogenannte Photoeffekt – die Emission von Elektronen aus einer Metalloberfläche durch Bestrahlung mit Licht – war 
	\section{Beispielrechnung}
Wähle z.\,B. $\lambda = \SI{0.071}{\nano\meter}$ und $\theta=90^\circ$. 
Setze \cref{eq:compton} ein und diskutiere die Größenordnung von $\Delta\lambda$.
		\section{Diskussion und Grenzen}
Thomson-Grenzfall ($\lambda \gg \lambda_C$), relativistische Korrekturen bei hohen Energien, 
experimentelle Bestätigungen und typische Fehlerquellen.
	\chapter{Photonen und die Zukunft der Physik}
\setcounter{section}{7}
\setcounter{subsection}{0}
\setcounter{subsubsection}{1}
\setcounter{secnumdepth}{3}


\subsection{Einleitung}
we
    \section{Fazit}
Knappe Zusammenfassung (ca. 150–200 Wörter): 
Was zeigt der Compton-Effekt über die Natur des Lichts? 
Relevanz für moderne Physik.
	\section*{Danksagung (optional)}
Kurzer Dank an Mitwirkende, Institutionen, Förderhinweise.


	
	% =============================================================
	% Literatur
	% =============================================================
	\renewcommand*{\bibfont}{\small}
	
	\printbibliography
	% =============================================================
	% Rückseite
	% =============================================================
	\clearpage
	\thispagestyle{empty}
	
	\vspace*{1.5cm}
	
	{\Large\bfseries Der Compton-Effekt}\\[0.5em]
	{\large\itshape Herleitung, Geschichte und Anwendungen}
	
	\vspace{1.2cm}
	
	\noindent
	\textbf{Über dieses Paper}\\
	Der Compton-Effekt markiert einen Wendepunkt in der Physik:
	Zum ersten Mal ließ sich eindeutig zeigen, dass Licht nicht
	nur Welleneigenschaften besitzt, sondern auch Impuls überträgt –
	wie ein Teilchen. Dieses Paper stellt die Herleitung der
	Compton-Formel verständlich und prägnant dar, ergänzt durch den
	historischen Kontext und die Bedeutung für die moderne Physik.
	
	\vspace{1.0cm}
	
	\noindent
	\textbf{Open Access}\\
	Die \emph{kostenlose} PDF-Version ist verfügbar unter:\\[2pt]
	\href{https://doi.org/10.5281/zenodo.xxxxxxx}{\texttt{https://doi.org/10.5281/zenodo.xxxxxxx}}
	
	\vspace{0.8cm}
	
	\noindent
	\textbf{Serie}\\
	\textit{Theoretische Physik – Herleitungen im historischen Kontext}\\
	Weitere Hefte und Materialien:\\
	\href{https://mathandphysics.de}{\texttt{mathandphysics.de}}
	
	\vfill
	
	\noindent
	\small
	\textit{Bibliografische Angaben}\\
	\PaperDOI\\
	\PaperISBNPDF\\
	\PaperISBNEPUB
	
	\vspace{0.8cm}
	
	\noindent
	\footnotesize
	© 2025 Christian Weilharter — Alle Rechte vorbehalten.\\
	Gedruckt auf Abruf über Amazon KDP. Satz: \LaTeX.
	
	\vfill
	\begin{center}
		\small
		\textit{Math \& Physics Paper Series – Open Access unter}\\[2pt]
		\href{https://mathandphysics.de}{\texttt{https://mathandphysics.de}} \, | \,
		\href{https://zenodo.org/communities/mathandphysics}{\texttt{Zenodo: Math \& Physics Open Science Collection}}
	\end{center}
	
\end{document}
