% =============================================================
% Paper Template (LaTeX) – "Derivations in Historical Context"
% Author: Christian Weilharter
% Series: Theoretical Physics – Derivations
% Purpose: 6×9 in (KDP/Print) + EPUB-compatible PDF (Apple Books)
% =============================================================
\documentclass[12pt,oneside]{scrartcl}
\usepackage[utf8]{inputenc}
\usepackage[T1]{fontenc}
\usepackage[english]{babel}
\usepackage{../styles/paper-style-en}

% ---- Literatur (BibLaTeX/Biber) ----
\usepackage[backend=biber,style=numeric,sorting=nyt]{biblatex}
\renewcommand*{\bibfont}{\small}
\addbibresource{../bib/literatur.bib}
% =============================================================
% Series and publication metadata
% =============================================================
\newcommand{\SeriesTitle}{Theoretical Physics – Derivations in Historical Context\par}
 
\newcommand{\PaperTitle}{\LARGE\textbf{Why the Photon Must Be (Almost) Massless}\\[0.4em]
	{\large From Gauge Symmetry to Astrophysical Limits\par}}



\newcommand{\PaperAuthor}{Christian Weilharter}
\newcommand{\PaperDate}{October 26, 2025}

\newcommand{\PaperDOI}{\emph{DOI (Zenodo)}: \texttt{10.5281/zenodo.xxxxxxx}}
\newcommand{\PaperISBNPDF}{\emph{ISBN (Print)}: 978-3-xxxx-xxxx-x}
\newcommand{\PaperISBNEPUB}{\emph{ISBN (E-Book)}: 978-3-xxxx-xxxx-x}
\newcommand{\PaperKeywords}{photon mass, gauge symmetry, Proca theory, quantum electrodynamics, astrophysical limits, photon helicity
}

% ---------- Metadata block ----------
\newcommand{\PaperMetadata}{
	\vspace{1em}
	\begin{flushleft}
		\small
		\setlength{\parskip}{3pt}
		\noindent\PaperDOI\\
	%	\PaperISBNPDF\\
		\PaperISBNEPUB\\
		\textit{Keywords:} \PaperKeywords
	\end{flushleft}
	\vspace{1em}
}

% =============================================================
% Title
% =============================================================


\begin{document}
	\thispagestyle{empty}
	
	\begin{titlepage}
	\begin{center}
		
	
		
		\vspace*{1.8cm}
		
		% --- TITLE ---
			\PaperTitle 
	
		% --- Series Title: viel näher dran! ---
		\vspace{0.6cm}
	
    %	{\large\textit\SeriesTitle }
		% \vspace{1.6cm}
		
		% --- AUTHOR ---
	
	    {\Large\PaperAuthor}
		\vspace{0.65cm}
		
		% --- DATE ---
		{\large\PaperDate}
		
		\vfill
	\end{center}
		% --- DOI / ISBN ---
	
		{\small
		
		
		\noindent\PaperDOI\\
	%	\PaperISBNPDF\\
	\PaperISBNEPUB\\
	\textit{Keywords:} \PaperKeywords
		
		}
		
	\end{titlepage}
	
	%\PaperMetadata
	
	\vspace{1em}
	\hrule
	\vspace{1em}
	
	\noindent\textbf{Abstract.}  
	
		This paper examines why the photon must be (almost) massless by combining 
		
	
	
	
	\vspace{1em}
	\hrule
	\vspace{2em}
	
	% =============================================================
	% Main content
	% =============================================================
	
	% Hauptkapitel
		\section{Introduction}

When Arthur Compton investigated the scattering of X-rays by electrons in 1923, he encountered a result that the classical wave theory of light could not explain. The wavelength of the scattered radiation was greater than that of the incident beam — and the difference depended on the scattering angle. This experiment provided the first direct evidence for the transfer of momentum between photon and electron, thereby confirming the particle nature of light.

The Compton effect marks a turning point in physics: it links energy and momentum conservation on a quantum-mechanical level and leads to the well-known \textbf{Compton formula}, which precisely describes the wavelength shift. The following derivation develops this relationship step by step — historically framed, mathematically sound, and physically transparent.

\vspace{1.2em}
\begin{quote}
	\small
	\textbf{Note on Methodology.}  
	This paper was created in close collaboration between human and artificial intelligence. 
	The physical reasoning and overall structure originate entirely from the author; the AI served as a tool for formulation, refinement, and didactic clarity. 
	The aim is to demonstrate that modern AI assistance does not replace human scientific work but enhances it — through greater clarity, structure, and transparency.
\end{quote}
\vspace{1.2em}
		\section{Historical Context – The Road to the Compton Effect}

\subsection{Initial Situation: Light as a Wave}

At the beginning of the 20th century, light was considered, according to the theory of \textbf{James Clerk Maxwell} (1860s), to be an electromagnetic wave. 
This model successfully explained interference, diffraction, and polarization – and appeared to be complete.


	\section{Derivation of the Compton Formula}
Energy and momentum conservation before and after the collision, elimination of the electron velocity, and conversion to wavelength form
	\chapter{Experimentelle Bestätigung des Photons}
\setcounter{section}{4}
\setcounter{subsection}{0}
\setcounter{subsubsection}{1}
\setcounter{secnumdepth}{3}
\setlength{\parindent}{0pt}
% Boxen-Stile definieren

\subsection{Der Photoeffekt }\index{Photoeffekt}

\subsubsection{ Einleitung und klassische Erwartung}\index{Klassische Elektrodynamik}\index{Wellenmodell}

Der sogenannte Photoeffekt – die Emission von Elektronen aus einer Metalloberfläche durch Bestrahlung mit Licht – war 
	\section{Beispielrechnung}
Wähle z.\,B. $\lambda = \SI{0.071}{\nano\meter}$ und $\theta=90^\circ$. 
Setze \cref{eq:compton} ein und diskutiere die Größenordnung von $\Delta\lambda$.
		\section{Diskussion und Grenzen}
Thomson-Grenzfall ($\lambda \gg \lambda_C$), relativistische Korrekturen bei hohen Energien, 
experimentelle Bestätigungen und typische Fehlerquellen.
	\chapter{Photonen und die Zukunft der Physik}
\setcounter{section}{7}
\setcounter{subsection}{0}
\setcounter{subsubsection}{1}
\setcounter{secnumdepth}{3}


\subsection{Einleitung}
we
	\section{Fazit}
Knappe Zusammenfassung (ca. 150–200 Wörter): 
Was zeigt der Compton-Effekt über die Natur des Lichts? 
Relevanz für moderne Physik.
	\section*{Danksagung (optional)}
Kurzer Dank an Mitwirkende, Institutionen, Förderhinweise.

	
	
	
	
	
	
	
	

	

	

	
	

	\section*{Acknowledgments}
	
	This work was created in collaboration with modern AI-based tools 
	(including ChatGPT) used as an intellectual co-pilot for structuring, 
	language refinement, and iterative development. All scientific 
	responsibility, physical reasoning, and final validation remain entirely 
	with the author.
	
	
	% =============================================================
	% References
	% =============================================================
%\section*{Literatur}
%\section*{References}
	\renewcommand*{\bibfont}{\small}
\printbibliography

	
	% ==========================================
	% Back page (final page) – revised, without repetition
	% ==========================================
	\clearpage
	\thispagestyle{empty}
	
	\vspace*{1.5cm}
	
	{\Large\bfseries Why the Photon Must Be (Almost) Massless}\\[0.5em]
%	{\large\itshape Derivation, History, and Applications}
	
	\vspace{1.2cm}
	
	\noindent
	\textbf{About this Paper}\\

	
	This work has examined the question of whether the photon could possess a 

	
	
	\noindent
	\textbf{Open Access}\\
	The \emph{free} PDF version is available at:\\[2pt]
	\href{https://doi.org/10.5281/zenodo.xxxxxxx}{\texttt{https://doi.org/10.5281/zenodo.xxxxxxx}}
	
	\vspace{0.8cm}
	
	\noindent
	\textbf{Series}\\
%	\textit{Theoretical Physics – Derivations in Historical Context}\\
	More papers and materials:\\
	\href{https://mathandphysics.eu}{\texttt{mathandphysics.eu}}
	
	\vfill
	
	\noindent
	\small
	\textit{Bibliographic Information}\\
	\PaperDOI\\
%	\PaperISBNPDF\\
	\PaperISBNEPUB
	
	\vspace{0.8cm}
	
	\noindent
	\footnotesize
	© 2025 Christian Weilharter — All rights reserved.\\
	Printed on demand via Amazon KDP. Typeset in \LaTeX.
	
	\vfill
	\begin{center}
		\small
		\textit{Math \& Physics Paper Series – Open Access at}\\[2pt]
		\href{https://mathandphysics.eu}{\texttt{https://mathandphysics.eu}} \, | \,
		\href{https://zenodo.org/communities/christian-copilot/records?q=&l=list&p=1&s=10&sort=newest}{\texttt{ Open Science Collection}}
	\end{center}
	
\end{document}
